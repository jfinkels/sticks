\documentclass{article}
\usepackage{aliascnt} % for correct autoref labeling of non-theorems
\usepackage[noend, boxed, linesnumbered]{algorithm2e}
\usepackage{amsfonts} % for \mathbb
\usepackage{amsmath} % for \implies
\usepackage{amsthm} % for theorems, definitions, lemmas, and styles
\usepackage{complexity}
\usepackage[
  backref=page, % backlinks in references
  pdftitle={On the computational complexity of the sticks problem},
  pdfauthor={Jeffrey Finkelstein}
]{hyperref} % for pdf links when rendering \ref and \cite commands

% don't print semicolons in pseudocode algorithms
\DontPrintSemicolon
\SetKwIF{If}{ElseIf}{Else}{if}{}{else if}{else}{end if}
\SetKwFor{ForAll}{for all}{}{end for all}

% Define theorem, lemma, and definition environments and corresponding styles.
% Lemmata, corollaries, and definitions are numbered with the same counter as
% that for theorems. We have to do some black magic to get the \autoref labels
% to work correctly.
\newtheorem{theorem}{Theorem}[section]

\newaliascnt{lemma}{theorem}
\newtheorem{lemma}[lemma]{Lemma}
\aliascntresetthe{lemma}

\newaliascnt{corollary}{theorem}
\newtheorem{corollary}[corollary]{Corollary}
\aliascntresetthe{corollary}

\newaliascnt{definition}{theorem}
\theoremstyle{definition} \newtheorem{definition}[definition]{Definition}
\aliascntresetthe{definition}

% define lemma, corollary, and definition context labels for \autoref command
% (theorem is already defined)
\newcommand{\lemmaname}{Lemma}
\newcommand{\corollaryname}{Corollary}
\newcommand{\definitionname}{Definition}

\newcommand{\dash}{\mbox{-}}
\newcommand{\defn}[1]{\emph{#1}}
\newcommand{\mor}{$\leq_m^p$}
\newcommand{\plain}[1]{\,\text{#1}\,}
\newcommand{\powerset}[1]{\mathcal{P}(#1)} % the power set operator
\newcommand{\triple}[3]{\langle#1,#2,#3\rangle} % generalization of pairing fn.

\newenvironment{langdef}[1]{\begin{definition}{\lang{#1}}}{\end{definition}}
\newenvironment{instance}{\\Instance:}{}
\newenvironment{question}{\\Question:}{}

\begin{document}

  \section{Variants of similar difficulty}

  \begin{langdef}{ASYM\dash STICKS}\label{def:asym-sticks}
    \begin{instance}
      a finite set of symbols $\Sigma$, a finite set of sticks
      $T\subseteq\Sigma^3$, and $k\in\mathbb{N}$
    \end{instance}
    \begin{question}
      Does there exist a layout $c\colon\{1,\ldots,k\}\to\Sigma$ such that for
        all tiles $(x, y, z)\in T$, there exists a position
        $n\in\{1,\ldots,k-2\}$ such that $c(n)=x$, $c(n+1)=y$ and
        $c(n+2)=z$?
    \end{question}
  \end{langdef}

  \begin{langdef}{SYM\dash STICKS}\label{def:sym-sticks}
    \begin{instance}
      a finite set of symbols $\Sigma$, a finite set of sticks
      $T\subseteq\Sigma^3$, and $k\in\mathbb{N}$
    \end{instance}
    \begin{question}
      Does there exist a layout $c\colon\{1,\ldots,k\}\to\Sigma$ such that for
        all tiles $(x, y, z)\in T$, there exists a position
        $n\in\{1,\ldots,k-2\}$ such that either $(c(n)=x$, $c(n+1)=y$ and
        $c(n+2)=z)$ or $(c(n)=z$, $c(n+1)=y$ and $c(n+2)=x)$?
    \end{question}
  \end{langdef}

  \begin{langdef}{CIRCULAR\dash ASYM\dash STICKS}\label{def:circular-asym-sticks}
    \begin{instance}
      a finite set of symbols $\Sigma$, a finite set of sticks
      $T\subseteq\Sigma^3$, and $k\in\mathbb{N}$
    \end{instance}
    \begin{question}
      Does there exist a layout $c\colon\{0,\ldots,k-1\}\to\Sigma$ such that
        for all tiles $(x, y, z)\in T$, there exists a position
        $n\in\{0,\ldots,k-1\}$ such that $c(n \plain{mod} k)=x$, $c(n+1
        \plain{mod} k)=y$ and $c(n+2 \plain{mod} k)=z$?
    \end{question}
  \end{langdef}

  \begin{langdef}{CIRCULAR\dash SYM\dash STICKS}\label{def:circular-sym-sticks}
    \begin{instance}
      a finite set of symbols $\Sigma$, a finite set of sticks
      $T\subseteq\Sigma^3$, and $k\in\mathbb{N}$
    \end{instance}
    \begin{question}
      Does there exist a layout $c\colon\{0,\ldots,k-1\}\to\Sigma$ such that
        for all tiles $(x, y, z)\in T$, there exists a position
        $n\in\{0,\ldots,k-1\}$ such that either $(c(n \plain{mod} k)=x$, $c(n+1
        \plain{mod} k)=y$ and $c(n+2 \plain{mod} k)=z)$ or $(c(n \plain{mod}
        k)=z$, $c(n+1 \plain{mod} k)=y$ and $c(n+2 \plain{mod} k)=x)$?
    \end{question}
  \end{langdef}

  \begin{lemma}
    \lang{ASYM\dash STICKS}\mor\lang{CIRCULAR\dash ASYM\dash STICKS}
  \end{lemma}
  \begin{proof}
    Construct transducer $f$ defined by $\triple{\Sigma}{T}{k} \mapsto
    \triple{\Sigma\cup\{x\}}{T\cup\{(x,x,x)\}}{k+3}$, where
    $x\notin\Sigma$. $f$ can be computed in polynomial time.

    Suppose $\triple{\Sigma}{T}{k}\in\lang{ASYM\dash STICKS}$. Then
    $f(\triple{\Sigma}{T}{k})$ outputs
    $\triple{\Sigma\cup\{x\}}{T\cup\{(x,x,x)\}}{k+3}$, as described
    above. Since $(x,x,x)$ can not overlap with any tile in $T$, it must be
    added without overlap to the layout for a penalty of $3$. Since the linear
    layout of $T$ has length at most $k$, the circular layout of
    $T\cup\{(x,x,x)\}$ has length at most $k+3$. Therefore
    $f(\triple{\Sigma}{T}{k})\in\lang{CIRCULAR\dash ASYM\dash STICKS}$.

    For the converse, suppose $\triple{\Sigma}{T}{k}\notin\lang{ASYM\dash
      STICKS}$. Then the shortest layout must have length at least $k+1$. And
    since $(x, x, x)$ cannot overlap with any tile in $T$, the layout of
    $f(\triple{\Sigma}{T}{k})$ on the circle must have length at least
    $(k+1)+3=k+4$, so $f(\triple{\Sigma}{T}{k})\notin\lang{CIRCULAR\dash
      ASYM\dash STICKS}$.

    Thus we have described a correct, polynomial time reduction from
    $\lang{ASYM\dash STICKS}$ to $\lang{CIRCULAR\dash ASYM\dash STICKS}$.
  \end{proof}

  \begin{lemma}
    \lang{SYM\dash STICKS}\mor\lang{CIRCULAR\dash SYM\dash STICKS}
  \end{lemma}
  \begin{proof}
  \end{proof}

  \section{\texorpdfstring{\NP}{NP}-complete variants}

  \begin{langdef}{ASYM\dash SET\dash STICKS}\label{def:asym-set-sticks}
    \begin{instance}
      a finite set of symbols $\Sigma$, a finite set of sticks
      $T\subseteq\powerset{\Sigma}^3$, and $k\in\mathbb{N}$
    \end{instance}
    \begin{question}
      Does there exist a layout $c\colon\{1,\ldots,k\}\to\Sigma$ such that for
        all tiles $(X, Y, Z)\in T$, there exists a position
        $n\in\{1,\ldots,k-2\}$ such that $c(n)\in X$, $c(n+1)\in Y$ and
        $c(n+2)\in Z$?
    \end{question}
  \end{langdef}

  \begin{definition}\label{def:hampath}
    Suppose $G=(V,E)$ is a directed graph, where $|V|=n$. $G$ contains a
    \defn{Hamiltonian path} if $\exists p\colon\{1,\ldots,n\}\to V$, a
    bijection, such that $\forall i\in\{1,\ldots,n-1\}$, $(p(i), p(i+1))\in E$.
  \end{definition}

  \begin{langdef}{HAMPATH}\label{def:hampathproblem}
    \begin{instance}
      a directed graph $G$
    \end{instance}
    \begin{question}
      Does $G$ contain a Hamiltonian path?
    \end{question}
  \end{langdef}

  \begin{langdef}{SYM\dash SET\dash STICKS}\label{def:sym-set-sticks}
    \begin{instance}
      a finite set of symbols $\Sigma$, a finite set of sticks
      $T\subseteq\powerset{\Sigma}^3$, and $k\in\mathbb{N}$
    \end{instance}
    \begin{question}
      Does there exist a layout $c\colon\{1,\ldots,k\}\to\Sigma$ such that for
        all tiles $(X, Y, Z)\in T$, there exists a position
        $n\in\{1,\ldots,k-2\}$ such that either $(c(n)\in X$, $c(n+1)\in Y$ and
        $c(n+2)\in Z)$ or $(c(n)\in Z$, $c(n+1)\in Y$ and $c(n+2)\in X)$?
    \end{question}
  \end{langdef}

  \begin{langdef}{UHAMPATH}\label{def:uhampath}
    \begin{instance}
      an undirected graph $G$
    \end{instance}
    \begin{question}
      Does $G$ contain a Hamiltonian path?
    \end{question}
  \end{langdef}

  % TODO reference
  \begin{lemma}\label{lem:hampath-npc}
    \lang{HAMPATH} is \NP-complete.
  \end{lemma}
  \begin{proof}
  \end{proof}

  \begin{theorem}\label{thm:asym-set-sticks-npc}
    \lang{ASYM\dash SET\dash STICKS} is \NP-complete.
  \end{theorem}
  \begin{proof}
  \end{proof}

  % TODO reference
  \begin{lemma}\label{lem:uhampath-npc}
    \lang{UHAMPATH} is \NP-complete.
  \end{lemma}
  \begin{proof}
  \end{proof}

  \begin{theorem}\label{thm:sym-set-sticks-npc}
    \lang{SYM\dash SET\dash STICKS} is \NP-complete.
  \end{theorem}
  \begin{proof}
  \end{proof}

  \begin{langdef}{ASYM\dash BIN\dash STICKS}\label{def:asym-bin-sticks}
    \begin{instance}
      a finite set of symbols $\Sigma$, a finite collection of bins $B_1, B_2,
      \ldots, B_N \subseteq \Sigma^3$, and $k\in\mathbb{N}$
    \end{instance}
    \begin{question}
      Does there exist a selection of one tile from each bin and a layout
        $c\colon\{1,\ldots,k\}\to\Sigma$ such that for all selected tiles $(x,
        y, z)$, there exists a position $n\in\{1,\ldots,k-2\}$ such that
        $c(n)=x$, $c(n+1)=y$ and $c(n+2)=z$?
    \end{question}
  \end{langdef}

  \begin{langdef}{SYM\dash BIN\dash STICKS}\label{def:sym-bin-sticks}
    \begin{instance}
      a finite set of symbols $\Sigma$, a finite collection of bins $B_1, B_2,
      \ldots, B_N \subseteq \Sigma^3$, and $k\in\mathbb{N}$
    \end{instance}
    \begin{question}
      Does there exist a selection of one tile from each bin and a layout
        $c\colon\{1,\ldots,k\}\to\Sigma$ such that for all selected tiles $(x,
        y, z)$, there exists a position $n\in\{1,\ldots,k-2\}$ such that either
        $(c(n)=x$, $c(n+1)=y$ and $c(n+2)=z)$ or $(c(n)=z$, $c(n+1)=y$ and
        $c(n+2)=x)$?
    \end{question}
  \end{langdef}

  \begin{theorem}\label{thm:asym-bin-sticks-npc}
    \lang{ASYM\dash BIN\dash STICKS} is \NP-complete.
  \end{theorem}
  \begin{proof}
    The reduction is from \lang{HAMPATH}. Define transducer $f$ as in
    \autoref{alg:hampath-to-asym-bin-sticks}.
    \begin{algorithm}\label{alg:hampath-to-asym-bin-sticks}
      \caption{Reduction from \lang{HAMPATH} to \lang{ASYM\dash BIN\dash
      STICKS}}
      \KwIn{$G=(V,E)$, where $|V|=n$}
      $\Sigma\gets V\cup E$\;
      $k\gets 2n+1$\;
      $B_v\gets\emptyset$, for all $v\in V$\;
      \ForAll{$v\in V$}{
        \If{$v$\,\textnormal{has at least one incoming edge and at least one
            outgoing edge}}{
          \hangindent 33pt
          $B_v\gets\{(e_{in},v,e_{out})|e_{in}, e_{out}\in E\plain{and}
          e_{in}=(u,v)\plain{and} e_{out}=(v,w)\plain{for some}u,w\in V\}$\;
        }
        \If{$v$\,\textnormal{has at least one incoming edge and no outgoing
            edges}}{
          $B_v\gets\{(e_{in},v,v)|e_{in}\in E\plain{and} e_{in}=(u,v)\plain{for
            some}u\in V\}$\;
        }
        \If{$v$\,\textnormal{has no incoming edges and at least one outgoing
            edge}}{
          $B_i\gets\{(v,v,e_{out})|e_{out}\in E\plain{and} e_{out}=(v,
          w)\plain{for some}w\in V\}$\;
        }
        \If{$v$\,\textnormal{has no incoming edges and no outgoing edges}}{
          $B_i\gets\{(v,v,v)\}$\;
        }
      }
      \Return{$\triple{\Sigma}{\{B_v\}_{v\in V}}{k}$}
    \end{algorithm}
    $f$ can be computed in polynomial time.
    
    Suppose $G=(V,E)\in\lang{HAMPATH}$, where $|V| = n$, so $\exists
    p\colon\{1,2,\ldots,n\}\to V$, a bijection, such that $\forall
    i\in\{1,2,\ldots,n-1\}$, $(p(i), p(i+1))\in E$. Then $f(G)$ outputs
    $\triple{\Sigma}{\{B_v\}_{v\in V}}{k}$ as described in
    \autoref{alg:hampath-to-asym-bin-sticks}. Then there is a layout of tiles
    chosen as follows: $\forall i\in\{2,3,\ldots,n-1\}$, choose from $B_{p(i)}$
    the tile $((p(i-1), p(i)), p(i), (p(i), p(i+1)))$. This tile exists in
    $B_{p(i)}$ because $(p(i-1),p(i))$ and $(p(i),p(i+1))$ are edges in $E$ by
    hypothesis. From $B_{p(1)}$ choose tile $(e_{start}, p(1), (p(1), p(2)))$,
    where $e_{start}\in E\cup\{p(1)\}$. From $B_{p(n)}$ choose tile $((p(n-1),
    p(n)), p(n), e_{end})$ where $e_{end}\in E\cup\{p(n)\}$. These two tiles
    must exist in these sets by hypothesis as well. Now the layout will be
    \begin{displaymath}
      e_{start}, p(1), (p(1), p(2)), p(2), (p(2), p(3)), \ldots, p(n-1),
      (p(n-1), p(n)), p(n), e_{end}
    \end{displaymath}

    For the converse, we suppose $f(G)\in\lang{ASYM\dash BIN\dash STICKS}$. If
    $f(G)$ outputs $\triple{\Sigma}{\{B_v\}_{v\in V}}{k}$, then there is a
    layout of one tile chosen from each $B_v$ such that the length of the
    layout is at most $k=2n + 1$. Since exactly one tile is chosen from each
    $B_v$, no tiles in any layout can overlap in two (or three) positions,
    because the middle component of each tile is unique (specifically, the
    middle component of each tile is a unique vertex in the graph
    $G$). Therefore the layout must have length at least $3n-(n-1)=2n+1$. Since
    the length of the layout is at least $2n+1$ and at most $2n+1$, it equals
    exactly $2n+1$. Therefore each tile overlaps the next one and the previous
    one in exactly one position on both the left and the right, except for the
    first and the last tiles which overlap only on the right and the left,
    respectively. Thus the layout has the form
    \begin{displaymath}
      e_{start}, v_1, (v_1, v_2), v_2, (v_2, v_3), \ldots, v_{n-1}, (v_{n-1},
      v_n), v_n, e_{end}
    \end{displaymath}
    We know the tiles must have this ``(edge, vertex, edge)'' format by the way
    we have designed $f$. Since each odd symbol in this layout is an edge in
    the graph $G$ (that is, excluding $e_{start}$ and $e_{end}$ which may not
    be), which connects the vertices immediately before and after it, reading
    each even symbol from left to right (that is, each of the vertices)
    describes a Hamiltonian path in $G$. Therefore, $G\in\lang{HAMPATH}$.

    Thus we have described a correct, polynomial time reduction from
    \lang{HAMPATH} to \lang{ASYM\dash BIN\dash STICKS}. Since \lang{HAMPATH}
    \NP-complete, it follows that \lang{ASYM\dash BIN\dash STICKS} is also
    \NP-complete.
  \end{proof}

  \begin{theorem}\label{thm:sym-bin-sticks-npc}
    \lang{SYM\dash BIN\dash STICKS} is \NP-complete.
  \end{theorem}
  \begin{proof}
  \end{proof}
\end{document}
