%%%%
%% sticks.tex
%%
%% Copyright 2010 Jeffrey Finkelstein
%%
%% Except where otherwise noted, this work is made available under the terms of
%% the Creative Commons Attribution-ShareAlike 3.0 license,
%% http://creativecommons.org/licenses/by-sa/3.0/.
%%
%% You are free:
%%    * to Share — to copy, distribute and transmit the work
%%    * to Remix — to adapt the work
%% Under the following conditions:
%%    * Attribution — You must attribute the work in the manner specified by
%%    the author or licensor (but not in any way that suggests that they
%%    endorse you or your use of the work).
%%    * Share Alike — If you alter, transform, or build upon this work, you may
%%    distribute the resulting work only under the same, similar or a 
%%    compatible license.
%%    * For any reuse or distribution, you must make clear to others the 
%%    license terms of this work. The best way to do this is with a link to the
%%    web page http://creativecommons.org/licenses/by-sa/3.0/.
%%    * Any of the above conditions can be waived if you get permission from
%%    the copyright holder.
%%    * Nothing in this license impairs or restricts the author's moral rights.
%%%%
\documentclass[draft]{article}
\usepackage{aliascnt} % for correct autoref labeling of non-theorems
\usepackage[noend, boxed, linesnumbered]{algorithm2e}
\usepackage{amsfonts} % for \mathbb
\usepackage{amsmath} % for \implies
\usepackage{amsthm} % for theorems, definitions, lemmas, and styles
\usepackage{complexity}
\usepackage{cclicenses}
\usepackage{gensymb} % for \degree

\usepackage[backref=page, pdftitle={On the computational complexity of the
  sticks problem}, pdfauthor={Jeffrey Finkelstein}]{hyperref}

% algorithm environment configuration
\DontPrintSemicolon
\SetKwIF{If}{ElseIf}{Else}{if}{}{else if}{else}{end if}
\SetKwFor{ForAll}{for all}{}{end for all}

% Define theorem, lemma, and definition environments and corresponding styles.
% Lemmata, corollaries, and definitions are numbered with the same counter as
% that for theorems. We have to do some black magic to get the \autoref labels
% to work correctly.
\newtheorem{theorem}{Theorem}[section]

\newaliascnt{lemma}{theorem}
\newtheorem{lemma}[lemma]{Lemma}
\aliascntresetthe{lemma}

\newaliascnt{corollary}{theorem}
\newtheorem{corollary}[corollary]{Corollary}
\aliascntresetthe{corollary}

\newaliascnt{proposition}{theorem}
\newtheorem{proposition}[proposition]{Proposition}
\aliascntresetthe{proposition}

\newaliascnt{definition}{theorem}
\theoremstyle{definition} \newtheorem{definition}[definition]{Definition}
\aliascntresetthe{definition}

% redefine footnote so it has no reference and no number
\long\def\symbolfootnote#1{\begingroup%
\def\thefootnote{\fnsymbol{footnote}}\footnotetext{#1}\endgroup} 

% for the creative commons license
\newcommand{\makelicense}{\begin{tabular}[h]{r l}
    \bysa & \parbox{275pt}{Copyright 2010 Jeffrey Finkelstein. 
      Except where otherwise noted, this work is licensed
      under http://creativecommons.org/licenses/by-sa/3.0/}
\end{tabular}}

% define lemma, corollary, and definition context labels for \autoref command
% (theorem is already defined)
\newcommand{\lemmaname}{Lemma}
\newcommand{\corollaryname}{Corollary}
\newcommand{\definitionname}{Definition}
\newcommand{\propositionname}{Proposition}
\newcommand{\algocflinename}{Algorithm}

\newcommand{\dash}{\mbox{-}}
\newcommand{\defn}[1]{\emph{#1}}
\newcommand{\mor}{\leq_m^p}
\newcommand{\moe}{\equiv_m^p}
\newcommand{\plain}[1]{\,\text{#1}\,}
\newcommand{\powerset}[1]{\mathcal{P}(#1)} % the power set operator
\newcommand{\triple}[3]{\langle#1,#2,#3\rangle} % generalization of pairing fn.

% language shortcuts
\newcommand{\asticks}{ASYM\dash STICKS}
\newcommand{\sticks}{SYM\dash STICKS}
\newcommand{\amsticks}{MULTI\dash ASYM\dash STICKS}
\newcommand{\msticks}{MULTI\dash SYM\dash STICKS}
\newcommand{\acsticks}{CIRCULAR\dash ASYM\dash STICKS}
\newcommand{\csticks}{CIRCULAR\dash SYM\dash STICKS}
\newcommand{\aisticks}{I\dash ASYM\dash STICKS}
\newcommand{\isticks}{I\dash SYM\dash STICKS}
\newcommand{\absticks}{BIN\dash ASYM\dash STICKS}
\newcommand{\bsticks}{BIN\dash SYM\dash STICKS}

\newenvironment{langdef}[1]{\begin{definition}{\lang{#1}}}{\end{definition}}
\newenvironment{instance}{\\Instance:}{}
\newenvironment{question}{\\Question:}{}

\author{Jeffrey Finkelstein}
\date{\today}

\begin{document}

  \symbolfootnote{\makelicense}

  \section{Definition}

  We will consider variants of the ``sticks'' problem, a game in which one
  attempts to determine the shortest one-dimensional layout of tiles, allowing
  overlaps for tiles with matching symbols. We will consider first the version
  in which each tile has three components---a left, a middle, and an end---with
  a symbol chosen from some alphabet on each component.

  \begin{langdef}{\asticks}\label{def:asym-sticks}
    \begin{instance}
      a finite set of symbols $\Sigma$, a finite set of sticks
      $T\subseteq\Sigma^3$, and $k\in\mathbb{N}$
    \end{instance}
    \begin{question}
      Does there exist a layout $c\colon\{1,\ldots,k\}\to\Sigma$ such that for
        all tiles $(x, y, z)\in T$, there exists a position
        $n\in\{1,\ldots,k-2\}$ such that $c(n)=x$, $c(n+1)=y$ and
        $c(n+2)=z$?
    \end{question}
  \end{langdef}

  We also define the symmetric version of the sticks problem, in which we are
  allowed to rotate tiles 180\degree.

  \begin{langdef}{\sticks}\label{def:sym-sticks}
    \begin{instance}
      a finite set of symbols $\Sigma$, a finite set of sticks
      $T\subseteq\Sigma^3$, and $k\in\mathbb{N}$
    \end{instance}
    \begin{question}
      Does there exist a layout $c\colon\{1,\ldots,k\}\to\Sigma$ such that for
        all tiles $(x, y, z)\in T$, there exists a position
        $n\in\{1,\ldots,k-2\}$ such that either $(c(n)=x$, $c(n+1)=y$ and
        $c(n+2)=z)$ or $(c(n)=z$, $c(n+1)=y$ and $c(n+2)=x)$?
    \end{question}
  \end{langdef}

  Note that it is always possible to determine a layout of length $3t$ for an
  instance of the sticks problem with $t$ tiles: simply lay them out in
  sequence without any overlaps.

  \section{Variants of similar difficulty}

  \begin{langdef}{\amsticks}
    \begin{instance}
      a finite set of symbols $\Sigma$, a finite \emph{multiset} of sticks
      $T\subseteq\Sigma^3$, and $k\in\mathbb{N}$
    \end{instance}
    \begin{question}
      Does there exist a layout $c\colon\{1,\ldots,k\}\to\Sigma$ such that for
        all tiles $(x, y, z)\in T$, there exists a position
        $n\in\{1,\ldots,k-2\}$ such that $c(n)=x$, $c(n+1)=y$ and
        $c(n+2)=z$?
    \end{question}
  \end{langdef}

  \begin{proposition}\label{prp:amsticks-to-asticks}
    $\lang{\amsticks}\mor\lang{\asticks}$
  \end{proposition}
  \begin{proof}
    Construct function $f$ defined by
    $\triple{\Sigma}{T}{k}\mapsto\triple{\Sigma}{T'}{k}$, where $T'$ is a set
    containing the elements of $T$ without repeats. $f$ can be computed in
    polynomial time.
    
    Suppose $\triple{\Sigma}{T}{k}\in\lang{\amsticks}$, so there is a layout of
    the tiles from $T$ of length at most $k$. For any repeated tile $t$, no
    matter the possible overlaps $t$ has in that layout with any other tile,
    removing it will either result in a layout of the same length (in case each
    component of the tile was overlapping some other tile) or a layout of
    smaller length, by simply removing the repeated tile, then sliding the
    remaining tiles on the left and right together. Therefore
    $f(\triple{\Sigma}{T}{k})$ has a layout of length at most $k$, so it is in
    $\lang{\asticks}$.

    For the converse, suppose $f(\triple{\Sigma}{T}{k})\in\lang{\asticks}$, so
    there is a layout of tiles from $T$, excluding the repeated tiles, of
    length at most $k$. By placing each identical tile on top of one another,
    we can achieve a layout with the same length, but using all the repeated
    tiles as well. Therefore, $\triple{\Sigma}{T}{k}\in\lang{\amsticks}$.

    Thus we have described a correct, polynomial time reduction from
    \lang{\amsticks} to \lang{\asticks}.
  \end{proof}

  \begin{corollary}
    $\lang{\asticks}\moe\lang{\amsticks}$
  \end{corollary}
  \begin{proof}
    By \autoref{prp:amsticks-to-asticks},
    $\lang{\amsticks}\mor\lang{\asticks}$, and
    $\lang{\asticks}\mor\lang{\amsticks}$ by the identity function. Therefore
    $\lang{\asticks}\moe\lang{\amsticks}$.
  \end{proof}

  Therefore it is sufficient to consider only the version of this problem
  without repeated tiles.

  Instead of considering a linear layout with two endpoints, we might consider
  the possibility of a layout along the circumference of a circle.

  \begin{langdef}{\acsticks}\label{def:circular-asym-sticks}
    \begin{instance}
      a finite set of symbols $\Sigma$, a finite set of sticks
      $T\subseteq\Sigma^3$, and $k\in\mathbb{N}$
    \end{instance}
    \begin{question}
      Does there exist a layout $c\colon\{0,\ldots,k-1\}\to\Sigma$ such that
        for all tiles $(x, y, z)\in T$, there exists a position
        $n\in\{0,\ldots,k-1\}$ such that $c(n \plain{mod} k)=x$, $c(n+1
        \plain{mod} k)=y$ and $c(n+2 \plain{mod} k)=z$?
    \end{question}
  \end{langdef}

  \begin{langdef}{\csticks}\label{def:circular-sym-sticks}
    \begin{instance}
      a finite set of symbols $\Sigma$, a finite set of sticks
      $T\subseteq\Sigma^3$, and $k\in\mathbb{N}$
    \end{instance}
    \begin{question}
      Does there exist a layout $c\colon\{0,\ldots,k-1\}\to\Sigma$ such that
        for all tiles $(x, y, z)\in T$, there exists a position
        $n\in\{0,\ldots,k-1\}$ such that either $(c(n \plain{mod} k)=x$, $c(n+1
        \plain{mod} k)=y$ and $c(n+2 \plain{mod} k)=z)$ or $(c(n \plain{mod}
        k)=z$, $c(n+1 \plain{mod} k)=y$ and $c(n+2 \plain{mod} k)=x)$?
    \end{question}
  \end{langdef}

  \begin{lemma}\label{lem:sticks-to-circ}
    $\lang{\asticks}\mor\lang{\acsticks}$
  \end{lemma}
  \begin{proof}
    Construct function $f$ defined by $\triple{\Sigma}{T}{k} \mapsto
    \triple{\Sigma\cup\{x\}}{T\cup\{(x,x,x)\}}{k+3}$, where
    $x\notin\Sigma$. $f$ can be computed in polynomial time.

    Suppose $\triple{\Sigma}{T}{k}\in\lang{\asticks}$. Then
    $f(\triple{\Sigma}{T}{k})$ outputs
    $\triple{\Sigma\cup\{x\}}{T\cup\{(x,x,x)\}}{k+3}$, as described
    above. Since $(x,x,x)$ can not overlap with any tile in $T$, it must be
    added without overlap to the layout for a penalty of $3$. Since the linear
    layout of $T$ has length at most $k$, the circular layout of
    $T\cup\{(x,x,x)\}$ has length at most $k+3$. Therefore
    $f(\triple{\Sigma}{T}{k})\in\lang{\acsticks}$.

    For the converse, suppose
    $\triple{\Sigma}{T}{k}\notin\lang{\asticks}$. Then the shortest layout must
    have length at least $k+1$. And since $(x, x, x)$ cannot overlap with any
    tile in $T$, the layout of $f(\triple{\Sigma}{T}{k})$ on the circle must
    have length at least $(k+1)+3=k+4$, so
    $f(\triple{\Sigma}{T}{k})\notin\lang{\acsticks}$.

    Thus we have described a correct, polynomial time reduction from
    $\lang{\asticks}$ to $\lang{\acsticks}$.
  \end{proof}

  \begin{lemma}
    $\lang{\sticks}\mor\lang{\csticks}$
  \end{lemma}
  \begin{proof}
    The proof of this lemma is the same as the proof of
    \autoref{lem:sticks-to-circ}, but we allow the possibility of rotating the
    tiles by 180\degree when choosing a layout.
  \end{proof}

  \section{\texorpdfstring{\NP}{NP}-complete variants}

  Recall the definitions of the following \NP-complete problems.

  \begin{definition}\label{def:hampath}
    Suppose $G=(V,E)$ is a directed graph, where $|V|=n$. $G$ contains a
    \defn{Hamiltonian path} if $\exists p\colon\{1,\ldots,n\}\to V$, a
    bijection, such that $\forall i\in\{1,\ldots,n-1\}$, $(p(i), p(i+1))\in E$.
  \end{definition}

  \begin{langdef}{HAMPATH}\label{def:hampathproblem}
    \begin{instance}
      a directed graph $G$
    \end{instance}
    \begin{question}
      Does $G$ contain a Hamiltonian path?
    \end{question}
  \end{langdef}

  \begin{langdef}{UHAMPATH}\label{def:uhampath}
    \begin{instance}
      an undirected graph $G$
    \end{instance}
    \begin{question}
      Does $G$ contain a Hamiltonian path?
    \end{question}
  \end{langdef}

  % TODO reference
  \begin{theorem}\label{thm:hampath-npc}
    \mbox{} % to put the next item on a new line
    \begin{enumerate}
      \renewcommand{\labelenumi}{\roman{enumi}.}
    \item \lang{HAMPATH} is \NP-complete
    \item \lang{UHAMPATH} is \NP-complete
    \end{enumerate}
  \end{theorem}

  Now we define generalizations of the \lang{\sticks} and \lang{\asticks}
  problems defined above, in which the three components of a tile are a finite
  set of symbols instead of a single symbol. The \lang{\isticks} and
  \lang{\aisticks} problems ask whether there is a layout of such tiles for
  which any overlap of tiles implies that the intersection of the two
  overlapping sets is not empty.

  \begin{langdef}{\aisticks}\label{def:asym-i-sticks}
    \begin{instance}
      a finite set of symbols $\Sigma$, a finite set of sticks
      $T\subseteq(\powerset{\Sigma}\backslash\emptyset)^3$, and
      $k\in\mathbb{N}$
    \end{instance}
    \begin{question}
      Does there exist a layout $c\colon\{1,\ldots,k\}\to\Sigma$ such that for
        all tiles $(X, Y, Z)\in T$, there exists a position
        $n\in\{1,\ldots,k-2\}$ such that $c(n)\in X$, $c(n+1)\in Y$ and
        $c(n+2)\in Z$?
    \end{question}
  \end{langdef}

  \begin{langdef}{\isticks}\label{def:sym-i-sticks}
    \begin{instance}
      a finite set of symbols $\Sigma$, a finite set of sticks
      $T\subseteq(\powerset{\Sigma}\backslash\emptyset)^3$, and
      $k\in\mathbb{N}$
    \end{instance}
    \begin{question}
      Does there exist a layout $c\colon\{1,\ldots,k\}\to\Sigma$ such that for
        all tiles $(X, Y, Z)\in T$, there exists a position
        $n\in\{1,\ldots,k-2\}$ such that either $(c(n)\in X$, $c(n+1)\in Y$ and
        $c(n+2)\in Z)$ or $(c(n)\in Z$, $c(n+1)\in Y$ and $c(n+2)\in X)$?
    \end{question}
  \end{langdef}

  \begin{theorem}\label{thm:asym-i-sticks-npc}
    \lang{\aisticks} is \NP-complete.
  \end{theorem}
  \begin{proof}
    The reduction is from \lang{HAMPATH}. Define function $f$ as in
    \autoref{alg:hampath-to-asym-i-sticks}.
    \begin{algorithm}\label{alg:hampath-to-asym-i-sticks}
      \caption{Reduction from \lang{HAMPATH} to \lang{\aisticks}}
      \KwIn{$G=(V,E)$, where $|V|=n$}
      $\Sigma\gets V\cup E$\;
      $k\gets 2n+1$\;
      Let $in(v)$ be the set of all incoming edges to $v$, $\forall v\in V$\;
      Let $out(v)$ be the set of all outgoing edges to $v$, $\forall v\in V$\;
      $T\gets\emptyset$\;
      \ForAll{$v\in V$}{
        \If{$v$\,\textnormal{has at least one incoming edge and at least one
            outgoing edge}}{
          $T\gets T\cup\{(in(v), v, out(v))\}$\;
        }
        \If{$v$\,\textnormal{has at least one incoming edge and no outgoing
            edges}}{
          $T\gets T\cup\{(in(v), v, v)\}$\;
        }
        \If{$v$\,\textnormal{has no incoming edges and at least one outgoing
            edge}}{
          $T\gets T\cup\{(v, v, out(v))\}$\;
        }
        \If{$v$\,\textnormal{has no incoming edges and no outgoing edges}}{
          $T\gets T\cup\{(v, v, v)\}$\;
        }
      }
      \Return{$\triple{\Sigma}{T}{k}$}
    \end{algorithm}
    $f$ can be computed in polynomial time.

    Suppose $G=(V,E)\in\lang{HAMPATH}$, where $|V| = n$, so $\exists
    p\colon\{1,2,\ldots,n\}\to V$, a bijection, such that $\forall
    i\in\{1,2,\ldots,n-1\}$, $(p(i), p(i+1))\in E$. Then $f(G)$ outputs
    $\triple{\Sigma}{T}{k}$ as described in
    \autoref{alg:hampath-to-asym-i-sticks}. Then there is a layout of tiles
    chosen as follows: the order of the internal (that is, not the first or the
    last) tiles in the layout will be $(in(p(i)), p(i), out(p(i)))$, $\forall
    i\in\{2,\ldots, n-1\}$, taken in increasing order. The first tile will be
    $(e_{start}, p(1), (p(1),p(2)))$, where $e_{start}\in E\cup\{p(1)\}$. The
    last tile will be $((p(n-1), p(n)), p(n), e_{end})$, where $e_{end}\in
    E\cup\{p(n)\}$. Since $(p(i),p(i+1))\in(out(p(i)\cap in(p(i+1)))$ for all
    $i\in\{1,2,\ldots,n-1\}$ by hypothesis, we can choose the following layout
    for the tiles:
    \begin{displaymath}
      e_{start}, p(1), (p(1), p(2)), p(2), (p(2), p(3)), \ldots, p(n-1),
      (p(n-1), p(n)), p(n), e_{end}
    \end{displaymath}
    The center of each tile will fall on a $p(i)$ for the appropriate $i$. The
    length of this layout is $2n+1$. Therefore $f(G)\in\lang{\aisticks}$.

    For the converse, suppose $f(G)\in\lang{\aisticks}$. If $f(G)$ outputs
    $\triple{\Sigma}{T}{k}$, then there is a layout of tiles in $T$ such that
    the length of the layout is at most $k=2n+1$. Since the middle component of
    each tile is a unique vertex in the graph $G$, no two tiles can overlap in
    two (or three) positions. Therefore the layout must have length at least
    $3n-(n-1)=2n+1$. Since the length of the layout is at least $2n+1$ and at
    most $2n+1$, it equals exactly $2n+1$. Therefore, each tile overlaps the
    next one and the previous one in exactly one position on both the left and
    the right, except for the first and the last tiles which overlap only on
    the right and left, respectively. Thus the layout has the form
    \begin{displaymath}
      e_{start}, v_1, (v_1, v_2), v_2, (v_2, v_3), \ldots, v_{n-1}, (v_{n-1},
      v_n), v_n, e_{end}
    \end{displaymath}
    We know the tiles must have a ``(set of edges, vertex, set of edges)''
    format by the way we have designed $f$. Note that by hypothesis, $(v_i,
    v_{i+1})\in out(v_i)\cap in(v_{i+1})$, for all
    $i\in\{1,2,\ldots,n-1\}$. Since each odd symbol in this layout is an edge
    in the graph $G$ (that is, excluding $e_{start}$ and $e_{end}$ which may
    not be) which connects the vertices immediately before and after it,
    reading each even symbol from left to right (that is, each of the vertices)
    describes a Hamiltonian path in $G$. Therefore, $G\in\lang{HAMPATH}$.

    Thus we have described a correct, polynomial time reduction from
    \lang{HAMPATH} to \lang{\aisticks}. Since \lang{HAMPATH} is \NP-complete,
    it follows that \lang{\aisticks} is also \NP-complete.
  \end{proof}

  The above proof in fact demonstrates that a more restrictive version of the
  \lang{\aisticks} problem is \NP-complete: the version which requires the
  middle component of each tile to be a single symbol chosen from the alphabet
  $\Sigma$ instead of an element in the powerset of $\Sigma$.

  \begin{theorem}\label{thm:sym-i-sticks-npc}
    \lang{\isticks} is \NP-complete.
  \end{theorem}
  \begin{proof}
  \end{proof}

  The \lang{\absticks} and \lang{\bsticks} problems defined below are a
  generalization of the \lang{\asticks} and \lang{\sticks} problems,
  respectively, in which one tile must be chosen from each of $N$ finite sets
  (``bins'') of tiles, in addition to choosing a layout for the chosen tiles.

  \begin{langdef}{\absticks}\label{def:asym-bin-sticks}
    \begin{instance}
      a finite set of symbols $\Sigma$, a finite collection of bins $B_1, B_2,
      \ldots, B_N \subseteq \Sigma^3$, and $k\in\mathbb{N}$
    \end{instance}
    \begin{question}
      Does there exist a selection of one tile from each bin and a layout
        $c\colon\{1,\ldots,k\}\to\Sigma$ such that for all selected tiles $(x,
        y, z)$, there exists a position $n\in\{1,\ldots,k-2\}$ such that
        $c(n)=x$, $c(n+1)=y$ and $c(n+2)=z$?
    \end{question}
  \end{langdef}

  \begin{langdef}{\bsticks}\label{def:sym-bin-sticks}
    \begin{instance}
      a finite set of symbols $\Sigma$, a finite collection of bins $B_1, B_2,
      \ldots, B_N \subseteq \Sigma^3$, and $k\in\mathbb{N}$
    \end{instance}
    \begin{question}
      Does there exist a selection of one tile from each bin and a layout
        $c\colon\{1,\ldots,k\}\to\Sigma$ such that for all selected tiles $(x,
        y, z)$, there exists a position $n\in\{1,\ldots,k-2\}$ such that either
        $(c(n)=x$, $c(n+1)=y$ and $c(n+2)=z)$ or $(c(n)=z$, $c(n+1)=y$ and
        $c(n+2)=x)$?
    \end{question}
  \end{langdef}

  The reduction from \lang{HAMPATH} to \lang{\absticks} is very similar to the
  reduction from \lang{HAMPATH} to \lang{\absticks} in the proof of
  \autoref{thm:asym-i-sticks-npc}.

  \begin{theorem}\label{thm:asym-bin-sticks-npc}
    \lang{\absticks} is \NP-complete.
  \end{theorem}
  \begin{proof}
    The reduction is from \lang{HAMPATH}. Define function $f$ as in
    \autoref{alg:hampath-to-asym-bin-sticks}.
    \begin{algorithm}\label{alg:hampath-to-asym-bin-sticks}
      \caption{Reduction from \lang{HAMPATH} to \lang{\absticks}}
      \KwIn{$G=(V,E)$, where $|V|=n$}
      $\Sigma\gets V\cup E$\;
      $k\gets 2n+1$\;
      $B_v\gets\emptyset$, for all $v\in V$\;
      \ForAll{$v\in V$}{
        \If{$v$\,\textnormal{has at least one incoming edge and at least one
            outgoing edge}}{
          \hangindent 33pt
          $B_v\gets\{(e_{in},v,e_{out})|e_{in}, e_{out}\in E\plain{and}
          e_{in}=(u,v)\plain{and} e_{out}=(v,w)\plain{for some}u,w\in V\}$\;
        }
        \If{$v$\,\textnormal{has at least one incoming edge and no outgoing
            edges}}{
          $B_v\gets\{(e_{in},v,v)|e_{in}\in E\plain{and} e_{in}=(u,v)\plain{for
            some}u\in V\}$\;
        }
        \If{$v$\,\textnormal{has no incoming edges and at least one outgoing
            edge}}{
          $B_i\gets\{(v,v,e_{out})|e_{out}\in E\plain{and} e_{out}=(v,
          w)\plain{for some}w\in V\}$\;
        }
        \If{$v$\,\textnormal{has no incoming edges and no outgoing edges}}{
          $B_i\gets\{(v,v,v)\}$\;
        }
      }
      \Return{$\triple{\Sigma}{\{B_v\}_{v\in V}}{k}$}
    \end{algorithm}
    $f$ can be computed in polynomial time.
    
    Suppose $G=(V,E)\in\lang{HAMPATH}$, where $|V| = n$, so $\exists
    p\colon\{1,2,\ldots,n\}\to V$, a bijection, such that $\forall
    i\in\{1,2,\ldots,n-1\}$, $(p(i), p(i+1))\in E$. Then $f(G)$ outputs
    $\triple{\Sigma}{\{B_v\}_{v\in V}}{k}$ as described in
    \autoref{alg:hampath-to-asym-bin-sticks}. Then there is a layout of tiles
    chosen as follows: $\forall i\in\{2,3,\ldots,n-1\}$, choose from $B_{p(i)}$
    the tile $((p(i-1), p(i)), p(i), (p(i), p(i+1)))$. This tile exists in
    $B_{p(i)}$ because $(p(i-1),p(i))$ and $(p(i),p(i+1))$ are edges in $E$ by
    hypothesis. From $B_{p(1)}$ choose tile $(e_{start}, p(1), (p(1), p(2)))$,
    where $e_{start}\in E\cup\{p(1)\}$. From $B_{p(n)}$ choose tile $((p(n-1),
    p(n)), p(n), e_{end})$ where $e_{end}\in E\cup\{p(n)\}$. These two tiles
    must exist in these sets by hypothesis as well. Now the layout will be
    \begin{displaymath}
      e_{start}, p(1), (p(1), p(2)), p(2), (p(2), p(3)), \ldots, p(n-1),
      (p(n-1), p(n)), p(n), e_{end}
    \end{displaymath}
    The length of this layout is $2n+1$. Therefore $f(G)\in\lang{\absticks}$.

    For the converse, we suppose $f(G)\in\lang{\absticks}$. If $f(G)$ outputs
    $\triple{\Sigma}{\{B_v\}_{v\in V}}{k}$, then there is a layout of one tile
    chosen from each $B_v$ such that the length of the layout is at most $k=2n
    + 1$. Since exactly one tile is chosen from each $B_v$, no tiles in any
    layout can overlap in two (or three) positions, because the middle
    component of each tile is unique (specifically, the middle component of
    each tile is a unique vertex in the graph $G$). Therefore the layout must
    have length at least $3n-(n-1)=2n+1$. Since the length of the layout is at
    least $2n+1$ and at most $2n+1$, it equals exactly $2n+1$. Therefore each
    tile overlaps the next one and the previous one in exactly one position on
    both the left and the right, except for the first and the last tiles which
    overlap only on the right and the left, respectively. Thus the layout has
    the form
    \begin{displaymath}
      e_{start}, v_1, (v_1, v_2), v_2, (v_2, v_3), \ldots, v_{n-1}, (v_{n-1},
      v_n), v_n, e_{end}
    \end{displaymath}
    We know the tiles must have this ``(edge, vertex, edge)'' format by the way
    we have designed $f$. Since each odd symbol in this layout is an edge in
    the graph $G$ (that is, excluding $e_{start}$ and $e_{end}$ which may not
    be) which connects the vertices immediately before and after it, reading
    each even symbol from left to right (that is, each of the vertices)
    describes a Hamiltonian path in $G$. Therefore, $G\in\lang{HAMPATH}$.

    Thus we have described a correct, polynomial time reduction from
    \lang{HAMPATH} to \lang{\absticks}. Since \lang{HAMPATH} is \NP-complete,
    it follows that \lang{\absticks} is also \NP-complete.
  \end{proof}

  \begin{theorem}\label{thm:sym-bin-sticks-npc}
    \lang{\bsticks} is \NP-complete.
  \end{theorem}
  \begin{proof}
  \end{proof}
\end{document}
