\documentclass{article}
\usepackage{aliascnt} % for correct autoref labeling of non-theorems
\usepackage[noend, % omit end of block tags
  noline, % omit vertical block lines
  linesnumbered, % number lines
  boxed, % put a box around the algorithm
]{algorithm2e} % for creating pseudocode
\usepackage{amsfonts} % for \mathbb
\usepackage{amsmath} % for \implies
\usepackage{amsthm} % for theorems, definitions, lemmas, and styles
\usepackage{complexity}
\usepackage[
  backref=page, % backlinks in references
  pdftitle={On the computational complexity of the sticks problem},
  pdfauthor={Jeffrey Finkelstein}
]{hyperref} % for pdf links when rendering \ref and \cite commands

% don't print semicolons in pseudocode algorithms
\DontPrintSemicolon
\SetKwIF{If}{ElseIf}{Else}{if}{}{else if}{else}{end if}%
\SetKwFor{ForEach}{for each}{}{end foreach}%

% Define theorem, lemma, and definition environments and corresponding styles.
% Lemmata, corollaries, and definitions are numbered with the same counter as
% that for theorems. We have to do some black magic to get the \autoref labels
% to work correctly.
\newtheorem{theorem}{Theorem}[section]

\newaliascnt{lemma}{theorem}
\newtheorem{lemma}[lemma]{Lemma}
\aliascntresetthe{lemma}

\newaliascnt{corollary}{theorem}
\newtheorem{corollary}[corollary]{Corollary}
\aliascntresetthe{corollary}

\newaliascnt{definition}{theorem}
\theoremstyle{definition} \newtheorem{definition}[definition]{Definition}
\aliascntresetthe{definition}

% define lemma, corollary, and definition context labels for \autoref command
% (theorem is already defined)
\newcommand{\lemmaname}{Lemma}
\newcommand{\corollaryname}{Corollary}
\newcommand{\definitionname}{Definition}

\newcommand{\dash}{\mbox{-}}
\newcommand{\defn}[1]{\emph{#1}}
\newcommand{\mor}{$\leq_m^p$}
\newcommand{\plain}[1]{\,\text{#1}\,}
\newcommand{\powerset}[1]{\mathcal{P}(#1)} % the power set operator
\newcommand{\triple}[3]{\langle#1,#2,#3\rangle} % generalization of pairing fn.

\newenvironment{langdef}[1]{\begin{definition}{\lang{#1}}}{\end{definition}}
\newenvironment{instance}{\\Instance:}{}
\newenvironment{question}{\\Question:}{}

\begin{document}

  \section{Variants of similar difficulty}

  \begin{langdef}{ASYM\dash STICKS}\label{def:asym-sticks}
    \begin{instance}
      a finite set of symbols $\Sigma$, a finite set of sticks
      $T\subseteq\Sigma^3$, and $k\in\mathbb{N}$
    \end{instance}
    \begin{question}
      Does there exist a layout $c\colon\{1,\ldots,k\}\to\Sigma$ such that for
        all tiles $(x, y, z)\in T$, there exists a position
        $n\in\{1,\ldots,k-2\}$ such that $c(n)=x$, $c(n+1)=y$ and
        $c(n+2)=z$?
    \end{question}
  \end{langdef}

  \begin{langdef}{SYM\dash STICKS}\label{def:sym-sticks}
    \begin{instance}
      a finite set of symbols $\Sigma$, a finite set of sticks
      $T\subseteq\Sigma^3$, and $k\in\mathbb{N}$
    \end{instance}
    \begin{question}
      Does there exist a layout $c\colon\{1,\ldots,k\}\to\Sigma$ such that for
        all tiles $(x, y, z)\in T$, there exists a position
        $n\in\{1,\ldots,k-2\}$ such that either $(c(n)=x$, $c(n+1)=y$ and
        $c(n+2)=z)$ or $(c(n)=z$, $c(n+1)=y$ and $c(n+2)=x)$?
    \end{question}
  \end{langdef}

  \begin{langdef}{CIRCULAR\dash ASYM\dash STICKS}\label{def:circular-asym-sticks}
    \begin{instance}
      a finite set of symbols $\Sigma$, a finite set of sticks
      $T\subseteq\Sigma^3$, and $k\in\mathbb{N}$
    \end{instance}
    \begin{question}
      Does there exist a layout $c\colon\{0,\ldots,k-1\}\to\Sigma$ such that
        for all tiles $(x, y, z)\in T$, there exists a position
        $n\in\{0,\ldots,k-1\}$ such that $c(n \plain{mod} k)=x$, $c(n+1
        \plain{mod} k)=y$ and $c(n+2 \plain{mod} k)=z$?
    \end{question}
  \end{langdef}

  \begin{langdef}{CIRCULAR\dash SYM\dash STICKS}\label{def:circular-sym-sticks}
    \begin{instance}
      a finite set of symbols $\Sigma$, a finite set of sticks
      $T\subseteq\Sigma^3$, and $k\in\mathbb{N}$
    \end{instance}
    \begin{question}
      Does there exist a layout $c\colon\{0,\ldots,k-1\}\to\Sigma$ such that
        for all tiles $(x, y, z)\in T$, there exists a position
        $n\in\{0,\ldots,k-1\}$ such that either $(c(n \plain{mod} k)=x$, $c(n+1
        \plain{mod} k)=y$ and $c(n+2 \plain{mod} k)=z)$ or $(c(n \plain{mod}
        k)=z$, $c(n+1 \plain{mod} k)=y$ and $c(n+2 \plain{mod} k)=x)$?
    \end{question}
  \end{langdef}

  \begin{lemma}
    \lang{ASYM\dash STICKS}\mor\lang{CIRCULAR\dash ASYM\dash STICKS}
  \end{lemma}
  \begin{proof}
  \end{proof}

  \begin{lemma}
    \lang{SYM\dash STICKS}\mor\lang{CIRCULAR\dash SYM\dash STICKS}
  \end{lemma}
  \begin{proof}
  \end{proof}

  \section{\NP-complete variants}

  \begin{langdef}{ASYM\dash SET\dash STICKS}\label{def:asym-set-sticks}
    \begin{instance}
      a finite set of symbols $\Sigma$, a finite set of sticks
      $T\subseteq\powerset{\Sigma}^3$, and $k\in\mathbb{N}$
    \end{instance}
    \begin{question}
      Does there exist a layout $c\colon\{1,\ldots,k\}\to\Sigma$ such that for
        all tiles $(X, Y, Z)\in T$, there exists a position
        $n\in\{1,\ldots,k-2\}$ such that $c(n)\in X$, $c(n+1)\in Y$ and
        $c(n+2)\in Z$?
    \end{question}
  \end{langdef}

  \begin{definition}\label{def:hampath}
    Suppose $G=(V,E)$ is a directed graph, where $|V|=n$. $G$ contains a
    \defn{Hamiltonian path} if $\exists p\colon\{1,\ldots,n\}\to V$, a
    bijection, such that $\forall i\in\{1,\ldots,n-1\}$, $(p(i), p(i+1))\in E$.
  \end{definition}

  \begin{langdef}{HAMPATH}\label{def:hampathproblem}
    \begin{instance}
      a directed graph $G$
    \end{instance}
    \begin{question}
      Does $G$ contain a Hamiltonian path?
    \end{question}
  \end{langdef}

  \begin{langdef}{SYM\dash SET\dash STICKS}\label{def:sym-set-sticks}
    \begin{instance}
      a finite set of symbols $\Sigma$, a finite set of sticks
      $T\subseteq\powerset{\Sigma}^3$, and $k\in\mathbb{N}$
    \end{instance}
    \begin{question}
      Does there exist a layout $c\colon\{1,\ldots,k\}\to\Sigma$ such that for
        all tiles $(X, Y, Z)\in T$, there exists a position
        $n\in\{1,\ldots,k-2\}$ such that either $(c(n)\in X$, $c(n+1)\in Y$ and
        $c(n+2)\in Z)$ or $(c(n)\in Z$, $c(n+1)\in Y$ and $c(n+2)\in X)$?
    \end{question}
  \end{langdef}

  \begin{langdef}{UHAMPATH}\label{def:uhampath}
    \begin{instance}
      an undirected graph $G$
    \end{instance}
    \begin{question}
      Does $G$ contain a Hamiltonian path?
    \end{question}
  \end{langdef}

  % TODO reference
  \begin{lemma}\label{lem:hampath-npc}
    \lang{HAMPATH} is \NP-complete.
  \end{lemma}
  \begin{proof}
  \end{proof}

  \begin{theorem}\label{thm:asym-set-sticks-npc}
    \lang{ASYM\dash SET\dash STICKS} is \NP-complete.
  \end{theorem}
  \begin{proof}
  \end{proof}

  % TODO reference
  \begin{lemma}\label{lem:uhampath-npc}
    \lang{UHAMPATH} is \NP-complete.
  \end{lemma}
  \begin{proof}
  \end{proof}

  \begin{theorem}\label{thm:sym-set-sticks-npc}
    \lang{SYM\dash SET\dash STICKS} is \NP-complete.
  \end{theorem}
  \begin{proof}
  \end{proof}

  \begin{langdef}{ASYM\dash BIN\dash STICKS}\label{def:asym-bin-sticks}
    \begin{instance}
      a finite set of symbols $\Sigma$, a finite collection of bins $B_1, B_2,
      \ldots, B_N \subseteq \Sigma^3$, and $k\in\mathbb{N}$
    \end{instance}
    \begin{question}
      Does there exist a selection of one tile from each bin and a layout
        $c\colon\{1,\ldots,k\}\to\Sigma$ such that for all selected tiles $(x,
        y, z)$, there exists a position $n\in\{1,\ldots,k-2\}$ such that
        $c(n)=x$, $c(n+1)=y$ and $c(n+2)=z$?
    \end{question}
  \end{langdef}

  \begin{langdef}{SYM\dash BIN\dash STICKS}\label{def:sym-bin-sticks}
    \begin{instance}
      a finite set of symbols $\Sigma$, a finite collection of bins $B_1, B_2,
      \ldots, B_N \subseteq \Sigma^3$, and $k\in\mathbb{N}$
    \end{instance}
    \begin{question}
      Does there exist a selection of one tile from each bin and a layout
        $c\colon\{1,\ldots,k\}\to\Sigma$ such that for all selected tiles $(x,
        y, z)$, there exists a position $n\in\{1,\ldots,k-2\}$ such that either
        $(c(n)=x$, $c(n+1)=y$ and $c(n+2)=z)$ or $(c(n)=z$, $c(n+1)=y$ and
        $c(n+2)=x)$?
    \end{question}
  \end{langdef}

  \begin{theorem}\label{thm:asym-bin-sticks-npc}
    \lang{ASYM\dash BIN\dash STICKS} is \NP-complete.
  \end{theorem}
  \begin{proof}
    The reduction is from \lang{HAMPATH}. Define transducer $f$ as in
    \autoref{fig:hampath-to-asym-bin-sticks}.
    \begin{algorithm}\label{fig:hampath-to-asym-bin-sticks}
      \caption{Reduction from \lang{HAMPATH} to \lang{ASYM\dash BIN\dash
      STICKS}}
      \KwIn{$G=(V,E)$, where $V=\{v_1,v_2,\ldots,v_n\}$}
      $\Sigma\gets V\cup E$\;
      $k\gets 2n+1$\;
      $B_i\gets\emptyset$, for all $i\in\{1,\ldots,n\}$\;
      \ForEach{$i\in\{1,\ldots,n\}$}{
        \If{$v_i$ has at least one incoming edge and at least one outgoing
          edge}{
          $B_i\gets\{(e_{in},v_i,e_{out})|e_{in}, e_{out}\in E\plain{and}
          e_{in}=(u,v_i)\plain{and} e_{out}=(v_i,w)\plain{for some}u,w\in
          V\}$\;
        }
        \If{$v_i$ has at least one incoming edge and no outgoing edges}{
          $B_i\gets\{(e_{in},v_i,v_i)|e_{in}\in E\plain{and}
          e_{in}=(u,v_i)\plain{for some}u\in V\}$\;
        }
        \If{$v_i$ has no incoming edges and at least one outgoing edge}{
          $B_i\gets\{(v_i,v_i,e_{out})|e_{out}\in E\plain{and}
          e_{out}=(v_i, w)\plain{for some}w\in V\}$\;
        }
        \If{$v_i$ has no incoming edges and no outgoing edges}{
          $B_i\gets\{(v_i,v_i,v_i)\}$\;
        }
      }
    \end{algorithm}
  \end{proof}

  \begin{theorem}\label{thm:sym-bin-sticks-npc}
    \lang{SYM\dash BIN\dash STICKS} is \NP-complete.
  \end{theorem}
  \begin{proof}
  \end{proof}
\end{document}
