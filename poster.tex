%%%%
%% poster.tex
%%
%% Copyright 2011 Jeffrey Finkelstein
%%
%% Except where otherwise noted, this work is made available under the terms of
%% the Creative Commons Attribution-ShareAlike 3.0 license,
%% http://creativecommons.org/licenses/by-sa/3.0/.
%%
%% You are free:
%%    * to Share — to copy, distribute and transmit the work
%%    * to Remix — to adapt the work
%% Under the following conditions:
%%    * Attribution — You must attribute the work in the manner specified by
%%    the author or licensor (but not in any way that suggests that they
%%    endorse you or your use of the work).
%%    * Share Alike — If you alter, transform, or build upon this work, you may
%%    distribute the resulting work only under the same, similar or a 
%%    compatible license.
%%    * For any reuse or distribution, you must make clear to others the 
%%    license terms of this work. The best way to do this is with a link to the
%%    web page http://creativecommons.org/licenses/by-sa/3.0/.
%%    * Any of the above conditions can be waived if you get permission from
%%    the copyright holder.
%%    * Nothing in this license impairs or restricts the author's moral rights.
%%%%
\documentclass{lposter}

\usepackage{amsthm}
\usepackage{complexity}

\title{Computational complexity of the linear matching systems problem}
\author{Jeffrey Finkelstein}
\advisor{Steve Homer}
\department{Computer Science}
%\institute{Boston University}
\date{\today}
\year{2011}

\newenvironment{instance}{\\Instance:}{}
\newenvironment{question}{\\Question:}{}

\theoremstyle{definition} \newtheorem*{definition}{Definition}

\begin{document}
\begin{poster}

\section{Motivation}

How does the human mind create and understand sentences in natural language?

How long would it take for a computer to use the same strategies?

\section{Problem definition}

Consider a linear tile with three cells. Each cell contains one color. Two
tiles can overlap wherever the cells have the same color. Given a set of tiles,
we wish to compute the linear layout of (possibly overlapping) tiles with the
shortest total length. Formally,


\begin{definition}
  Instance: a finite set of symbols $\Sigma$, a finite set of sticks
  $T\subseteq\Sigma^3$, and $k\in\mathbb{N}$

  Question: Does there exist a layout $c\colon\{1,\ldots,k\}\to\Sigma$ such
  that for all tiles $(x, y, z)\in T$, there exists a position
  $n\in\{1,\ldots,k-2\}$ such that $c(n)=x$, $c(n+1)=y$ and $c(n+2)=z$?
\end{definition}

\section{Example instance}

\section{Complexity}

\section{NP-complete variants}

\section{Further work}

\bibliography{sample}

\end{poster}

%% \begin{frame}{}
%%   \begin{columns}[t]

%%     %% left column
%%     \begin{column}{.3\linewidth}
%%       \begin{block}{\VeryHuge Motivation}
%%         \begin{itemize}\veryHuge
%%         \item foo
%%         \item bar
%%         \end{itemize}
%%       \end{block}
%%       %\vspace{2in}
%%       \begin{block}{\VeryHuge Problem definition}
%%         \begin{itemize}\veryHuge
%%         \item foo
%%         \item bar
%%         \end{itemize}
%%       \end{block}
%%     \end{column}

%%     %% middle column
%%     \begin{column}{.3\linewidth}
%%       \begin{block}{\VeryHuge Example instance}
%%         %\includegraphics{images/rbc_gi.png}
%%       \end{block}
%%       %\vspace{0.2in}
%%       \begin{block}{\VeryHuge Complexity}
%%         \begin{itemize}\veryHuge
%%         \item foo
%%         \item bar
%%         \end{itemize}
%%       \end{block}
%%     \end{column}

%%     %% right column
%%     \begin{column}{.3\linewidth}
%%       \begin{block}{\VeryHuge $\NP$-complete variants}
%%         \begin{itemize}\veryHuge
%%         \item foo
%%         \item bar
%%         \end{itemize}
%%       \end{block}
%%       %\vspace{1.8in}
%%       \begin{block}{\VeryHuge Further problems}
%%         \begin{itemize}\veryHuge
%%         \item foo
%%         \item bar
%%         \end{itemize}
%%       \end{block}
%%     \end{column}
%%   \end{columns}
%% \end{frame}
\end{document}
