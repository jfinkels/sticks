The following variants of the LMT problem are \NP-complete.

{\bf LMT with intersecting sets}. If we allow each tile
to have a set of colors and allow tiles to overlap if the intersection of the
overlapping sets is non-empty, then the problem is \NP-complete.

\colorlet{shadecolor}{red!40!gray!20}
\begin{shaded}
\begin{proof}[Proof sketch]
  The reduction is from $\lang{HAMPATH}$. Transform each vertex of the graph
  into a tile, where the middle cell is the vertex, the left cell is the set of
  all incoming edges, and the right cell is the set of all outgoing edges. Then
  there is a layout of length $3|V| - (|V| - 1) = 2|V| + 1$ if and only if
  there is a Hamiltonian path in $G$.
\end{proof}
\end{shaded}

The original LMT problem is (basically) the restriction of this variant to sets
of tiles containing only singleton sets.

{\bf LMT chosen from bins}. If we are given a set of bins
containing tiles and require that one must be chosen from each bin, then the
problem is \NP-complete.

\begin{shaded}
\begin{proof}[Proof sketch]
  The reduction is again from $\lang{HAMPATH}$. For each vertex, create one bin
  and place in it one tile for each pair of incoming and outgoing edge, so that
  the left cell of the tile is one incoming edge, the middle cell is the
  vertex, and the right cell is one outgoing edge. Then there is a layout of
  length $2|V| + 1$ with one cell chosen from each bin if and only if there is
  a Hamiltonian path in $G$.
\end{proof}
\end{shaded}

The original LMT problem is (basically) the restriction of this variant to sets
of bins containing exactly one tile.

That the original problem is a restriction of these \NP-complete variants may
suggest that it is computationally \emph{easier}, though we can not currently
prove this.
